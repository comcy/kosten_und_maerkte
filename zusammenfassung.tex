\documentclass[a4paper,11pt, twoside]{article}

%%%%%%%%%%%%%%%%%%%%%%%%%%%%%
%% Packages
% System - Sprache
\usepackage[utf8x]{inputenc}
\usepackage[ngerman]{babel}
\usepackage[T1]{fontenc}
\usepackage{lmodern}
\usepackage{ucs}
\PrerenderUnicode{äüößÄÜÖ} 

\usepackage[colorlinks=true, pdfborder={0 0 0}, pdftex, breaklinks=true, linkcolor=blue, citecolor=orange , filecolor=purple , urlcolor=purple, linktocpage=true]{hyperref}

% Seiten Layout
\usepackage{fancyhdr}
\usepackage[left=2cm,right=3cm,top=2cm,bottom=2cm]{geometry}

% TIKZ
\usepackage{tikz}
\usepackage{pgf}

% Inhalt
\usepackage{wrapfig} %floatfig
\usepackage{sidecap} %floatfig
\usepackage{float}
\usepackage{dpfloat}
\usepackage{algorithmicx}
\usepackage{algorithm}
\usepackage{algpseudocode}
\usepackage{prettyref}
\usepackage{titleref}
\usepackage{listings}
\usepackage{glossaries}
\usepackage{nomencl}
\usepackage{cite}
\usepackage{graphicx}
\usepackage{verbatim}
\usepackage[font=small,labelfont=bf]{caption}

% Mathe
\usepackage{calc}
\usepackage{ifthen}
\usepackage{times}
\usepackage{amsmath}
\usepackage{mathtools}

% hyperref
\usepackage[nohyperlinks%,printonlyused, 
]{acronym}

\newcommand\mpar[1]{\marginpar {\flushleft\small #1}}
\setlength{\marginparwidth}{2cm}

%%%%%%%%%%%%%%%%%%%%%%%%%%%%%
%% Farben
\definecolor{lightgrey}{gray}{.8}
\definecolor{lila}{rgb}{0.,0,0.5}

%%%%%%%%%%%%%%%%%%%%%%%%%%%%%
%% Farben
\definecolor{lightgrey}{gray}{.8}
\definecolor{lila}{rgb}{0.,0,0.5}

\begin{document}

\title{
\textbf{Kosten und Märkte}\\
Zusammenfassung
}

\author{Christian Silfang}
\date{Somersemester 2014}

\parskip1.5ex
\parindent0em

\pagestyle{empty}
\maketitle
\thispagestyle{empty}
\cleardoublepage

%\noindent\rule[1ex]{\textwidth}{1pt}

%\vspace{1cm}
\tableofcontents
\cleardoublepage

\pagestyle{fancy}
	\renewcommand{\sectionmark}[1]{\markboth{#1}{}}
	\fancyhf{}
	\fancyhead[EL]{\thesection { }\leftmark}
	\fancyhead[OR]{{ }\rightmark}
	\renewcommand{\headrulewidth}{0.4pt}
	\pagenumbering{arabic}
	
  \fancyfoot[EL]{\textbf{\thepage}}
	\fancyfoot[OR]{\textbf{\thepage}} 
	
	\setcounter{page}{1}

\section{Planung \& Kontrolle}

\subsection{Grundbegriffe}

\subsubsection*{Eigenschaften von Strategien}
\begin{itemize}
	\item legen Aktivitätsfelder fest, sind kokurrenzbezogen
	\item sind konkurrenzbezogen
	\item nehmen Bezug auf die Umweltsituation und –entwicklung (Chancen und Bedrohungen)
	\item beziehen sich auf Unternehmensressourcen relativ zur Konkurrenz
	\item zeigen Einstellungen/Wertvorstellungen der Entscheidungsträger
	\item auf gesamtes Geschäftsfeld ausgerichtet
	\item hohe Bedeutung für Vermögens-/Ertragslage des Unternehmens
	\item weitreichende Konsequenzen
	\item zukunftsorientiert
	\item können (!) einem systematischen Planungsprozeß entspringen
\end{itemize}

\subsubsection*{Leitfragen}
\begin{enumerate}
	\item Tätigkeit in welchen Geschäftsfeldern?
	\item Wie soll Wettbewerb in Geschäftsfeldern bestritten werden?
	\item Was ist die längerfristige Erfolgsbasis (Kernkompetenz)?
\end{enumerate}

\textbf{Gesamtunternehmen:} Gesamtunternehmensstrategie (Corporate Strategy)\\
\textbf{Geschäftsfeld:} Wettbewerbsstrategie (Business Strategy)

\subsection{Umweltanalyse}

Umweltanalyse ist Kernstück der strategischen Analyse und ermittelt Chancen/Bedrohungen

\subsubsection*{Betrachtungen}
$\rightarrow$ Wettbewerbsumfeld/Geschäftsfeld: Analyse der Branchenstruktur nach \footnote{
%% Fußnote
\textbf{Five-Forces von Porter}: Markterfolg hängt im wesentlichen von Marktstruktur ab (Seite 28/Abbildung):
\begin{enumerate}
	\item Wettbewerber einer Branche (Rivalen)
	\item Potenzielle neue Anbieter (Bedrohung)
	\item Ersatzprodukteb(Substitutionsgefahr)
	\item Lieferanten (Verhandlungsstärke)
	\item Abnehmer (Verhandlungsmacht)
\end{enumerate}
}{\textit{Porter}}

\texttt{GRAFIK}

$\rightarrow$ Betrachtet allgemeine Umwelt:
\begin{itemize}
	\item Makroökonomische Umwelt
	\begin{itemize}
		\item Konjunkturentwicklung, Wechselkurse, Entwicklung des Arbeitsmarktes, wirtschaftliche Entwicklung (global/nach Region)
	\end{itemize}
	\item Technologische Umwelt
	\begin{itemize}
		\item Entwicklung der Technologie als wesentlicher Treiber, S-Kurven-Modell (Technologielebenszyklus)
	\end{itemize}
	\item Politisch-rechtliche Umwelt
	\begin{itemize}
		\item politische Entwicklung auf allen Ebenen, Zölle/Subventionen
		\item internationale Tendenzen (Verschuldung, 3. Welt, Kyoto, Osteuropa), Krisen
	\end{itemize}
	\item Soziokulturelle Umwelt
	\begin{itemize}
		\item Demographische Entwicklung, Wertewandel
	\end{itemize}
	\item Natürliche Umwelt
	\begin{itemize}
		\item Benötigte Ressourcen (Reichweite/Verteilung), Entsorgung
	\end{itemize}
\end{itemize}

\subsubsection*{Vorgehensweise}
Bestimmung von relevanten Schlüsselgrößen und Prognosen über deren Entwicklung. Analyse von Querverbindungen über Entwurf/Bewertung von alternativen Szenarien. Festellung der Prämisse für weitere Planungsprozesse.

\subsection{Unternehmensanalyse}

Ermittlung der eigenen Stärken und Schwächen, dazu sind zwei Sichtweisen erforderlich:
\begin{enumerate}
	\item \textbf{Wertschöpfungssicht:} eigene Stärken/Schwähen relativ zur Konkurrenz
	\item \textbf{Kundensicht:} kritische Erfolgsfaktoren aus Sicht des Marktes, eigenes Profil vs. Profil der Wettbewerber
\end{enumerate}
$\rightarrow$ beide Sichtweisen ergeben \textbf{Potentiale und Wettbewerbsvorteile}(Vgl. Wertkette nach \textit{Porter})
	%% Randnotiz	
	\mpar{\textcolor{red}{Abschätzung der eigenen preislichen Lage auf dem Markt}}
\texttt{GRAFIK}

Erfolgsfaktoren können in verschiedene Faktoren eingeteilt werden:
\begin{itemize}
	\item Finanzielle
	\item Physische $\rightarrow$ häufiger Erfolgsfaktor
	\item Humane $\rightarrow$ häufiger Erfolgsfaktor
	%% Randnotiz	
	\mpar{\textcolor{red}{Monopolstellung bei vorhanden Ressourcen ist immer wertvoll!}}
	\item Organisatorische
	\item Technologische
	\item Finanzielle
\end{itemize}
%% Randnotiz	
\mpar{\textcolor{red}{Ressourcen die jeder hat, sind nicht wertvoll!}}

\subsubsection*{Merkmale "`strategischer"' Ressourcen}
\begin{itemize}
	\item Einmaligkeit: knappe Ressourcen, monopolähnlicher Zugang  
	\item Eingeschränkte Imitierbarkeit: Zusammenhänge schwer erkennbar, historisch gewachsene Ressourcen/Situationen, soziale Komplexität	
	\item Fehlende Substituierbarkeit: Ressourcen sind nicht durch andere ersetzbar
	\item Wert für die Strategie: Ressourcen müssen gewinnbringend zur Steigerung der Wettbewerbsfähigkeit verwendet werden können
\end{itemize}

\subsubsection*{Möglichkeiten zur Selbstreflektion}
%% Randnotiz	
\mpar{\textcolor{red}{Erhebungen durch Kundenbefragung ODER Vergleich mit Konkurrenz}}
\[ \left.
\begin{array}{l l}
\text{\textbf{S}} & \quad \text{trength}\\
\text{\textbf{W}} & \quad \text{eakness}\\
\text{\textbf{O}} & \quad \text{portunities}\\
\text{\textbf{T}} & \quad \text{reatments}
\end{array} 
\right\} \to \text{SWOT-Analyse} \quad \quad	\begin{array}{c|c}
					  	 S & W \\
					  	 \hline
					  	 O & T
						\end{array} \]

Teilweise Betrachtung aus Sicht des Kunden, dazu Ermittlung kritischer Erfolgsfaktoren wie Qualität, Service, Flexibilität, Termintreue und Preis.

\subsection{Strategische Optionen}
\texttt{GRAFIK}\\
%% Randnotiz	
\mpar{\textcolor{red}{Nischenmarkt: schwierig, falls besetzt}}
Es existieren \underline{drei} zentrale Fragen zur Ermittlung der strategischen Optionen:
\begin{enumerate}
	\item Wo soll konkurriert werden? (Ort des Wettbewerbs)
	\item Wie soll konkurriert werden? (Regeln des Wettbewerbs: Rabatte, Preisnachlass)
	\item Mit welcher Stoßrichtung soll konkurriert werden? (Schwerpunkt des Wettbewerbs: Massen- oder Billigproduktion)
\end{enumerate}

Überlegungen nur sinnvoll wenn Unternehmen in mehreren Geschäftsfeldern tätig ist oder eine Erweiterung auf mehrere geschäftsfelder geplant wird. 
Die möglichen Optionen sind:

\subsubsection*{Diversifikation} 
\texttt{GRAFIK}

\subsubsection*{Portfolio-Strategien} 
\texttt{GRAFIK}

\subsubsection*{Internationalisierung} 

\subsubsection*{Kernkompetenzorientierung} 


\end{document} 
